\documentclass[journal]{IEEEtran}
\usepackage[utf8]{inputenc}

\usepackage{graphicx}
%\usepackage[caption=false,font=footnotesize]{subfig}
\usepackage{url}
\usepackage{algorithmic}%
\usepackage{algorithm}%
\usepackage{multirow}%
\usepackage{xcolor}%
\usepackage{subcaption}%
\usepackage{placeins}%
\usepackage{circuitikz}%

\usepackage{url}
% Making the references and links clickable
\usepackage{hyperref}
\hypersetup{
    %colorlinks=false,
    pdfborder={0 0 0}
}

\usepackage{xcolor}
\newcommand{\toDo}[1]{\textcolor{red}{#1}}
\newcommand{\mymotor}[2] % #1 = name , #2 = rotation angle
{\draw[thick,rotate=#2] (#1) circle (10pt)
 node[]{$\mathsf M$} 
++(-12pt,3pt)--++(0,-6pt) --++(2.5pt,0) ++(-2.8pt,6pt)-- ++(2.5pt,0pt);
\draw[thick,rotate=#2] (#1) ++(12pt,3pt)--++(0,-6pt) --++(-2.5pt,0) 
++(2.8pt,6pt)-- ++(-2.5pt,0pt);
}

\usepackage{fancyhdr}
\pagestyle{fancy}
\lhead{University of Brasília}
\rhead{\thepage}
\cfoot{Comparing the Performance of Finite-State Machines with Different Numbers of States on TORCS}
\renewcommand{\headrulewidth}{0.4pt}
\renewcommand{\footrulewidth}{0.4pt}

\begin{document}
    \title{Comparing the Performance of\\Finite-State Machines with\\Different Numbers of States on TORCS}
    
    \author{\IEEEauthorblockN{Bruno H. F. Macedo and
            Gabriel F. P. Araujo}
        \IEEEauthorblockA{
        % \\Laboratório de Automação e Robótica - LARA
        \\Universidade de Brasília - UnB\\
        Brasília - DF - Brasil}
        }
    
    \maketitle
    
    \begin{abstract}
        
        This work presents the design and implementation of an H bridge. The circuit was designed to be used with low power DC motors. This report will present as well the design, prototyping and manufacturing of the board.
                
    \end{abstract}
    
    \begin{IEEEkeywords}
        H bridge, robotics, electric circuits, electronic circuits.        
    \end{IEEEkeywords}
    
 
    \section{\textbf{Introduction}}\label{sec:1}
	

	H bridges (see Figure~\ref{fig:bridge}) are electronic circuits used to enable a voltage to be aplied across a motor in either direction. These motor are often used in robotics to able a motor to run forward and backward.

\begin{figure*}[t]
\centering
% \begin{subfigure}[b]{0.4\textwidth}
	\centering%
	\includegraphics[height=.25\textwidth]{img/H_bridge.png}
	\caption{H Bridge figure borrowed from wikipedia~\cite{WIKI}.}\label{fig:bridge}%
% \end{subfigure}\hfill
\end{figure*}

\begin{figure}[htp!]
\begin{center}
\begin{circuitikz} %\label{sec:1}
    \draw (2,0) node[npn](q1) at (2,0){}
    (q1.E) node[npn](q4) at (2,-1.55){}
    (q1.E) -- (q4.C);
    \draw (-0.55,0) to[R] (q1.B){};
    \draw (-0.55,-1.55) to[R] (q4.B){};
    \draw (5,0) node[npn,xscale=-1](q3) at (4,0){}
    (q3.E) node[npn,xscale=-1](q2) at (4,-1.55){}
    (q3.E) -- (q2.C)
    (q1.C) -- (q3.C)
    (q4.E) -- (q2.E);
    \draw (3,-2.55) node[ground](gnd){}
    (q2.E) |- (gnd);
    \draw (q3.B)  to[R] (6,0);
    \draw (q2.B) to[R] (6,-1.55);
    % \draw (q1.E) node[elmech]{M}) (q3.E);
    \draw (-0.55,0) -- (-1.0,0) node[circ]{} -- (-1.0,-4) -- (7,-4) -- (7,-1.55) -- (6,-1.55);
    \draw (-0.55,-1.55) -- (-0.80,-1.55) -- (-0.8,-3.5) --(6.5,-3.5) -- (6.5,0) node[circ]{} -- (6,0);
    \draw (3,2) node[spdt,xscale=-1] (s1){}
    (s1.in) |- node[circ]{} (q1.C)
    (s1.out 1) |- (3.5,2.5) |- (7,2.5) |- (6,0)
    (s1.out 2) |- (-1,1.685) |- (-1,0);
    \draw (q1.E) to[sV, color=white, name=M1] (q3.E);
    \mymotor{M1}{0}
    \draw (q2.E) node[circ]{} |- (7.5,-2.31) to[battery] (7.5,1) |- (4,1) |- node[circ]{} (q3.C);
    % \draw (q1.C) --++(0,0.5) node[vcc]{+4.5\,\textnormal{V}};
    % \mymotor{M}{0}
    % \draw (q1.B) -- to[R=100<\ohm>]{};
    % \draw (0,0) node[npn](npn) at (0,0){};
\end{circuitikz}
\caption{Ponte H}\label{fig:schem}%
\end{center}
\end{figure}


    % \input{TORCS_Environment}
    % \input{Related_Works}
    % \input{FSMDriver_ControllerStructure}
    % \input{Genetic_Algorithm}
    % \input{Methodology_Experiments}
    % \input{Analysis_Conclusion}
    % \input{Future_Works}
    
    %\nocite{*}
    \bibliographystyle{IEEEtran}
    \bibliography{Bibliography}
    \toDo{Adequar as referências ao padrão do IEEE: \url{http://www.ieee.org/documents/ieeecitationref.pdf}. Quando
    há 3 ou mais autores colocar et al.}
    
    
    
\end{document}
