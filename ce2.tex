\documentclass[journal]{IEEEtran}
\usepackage[utf8]{inputenc}

\usepackage{graphicx}
%\usepackage[caption=false,font=footnotesize]{subfig}
\usepackage{url}
\usepackage{algorithmic}%
\usepackage{algorithm}%
\usepackage{multirow}%
\usepackage{xcolor}%
\usepackage{subcaption}%
\usepackage{placeins}%
\usepackage{circuitikz}%

\usepackage{url}
% Making the references and links clickable
\usepackage{hyperref}
\hypersetup{
    %colorlinks=false,
    pdfborder={0 0 0}
}

\usepackage{xcolor}
\newcommand{\todo}[1]{\textcolor{red}{#1}}
\newcommand{\mymotor}[2] % #1 = name , #2 = rotation angle
{\draw[thick,rotate=#2] (#1) circle (10pt)
 node[]{$\mathsf M$} 
++(-12pt,3pt)--++(0,-6pt) --++(2.5pt,0) ++(-2.8pt,6pt)-- ++(2.5pt,0pt);
\draw[thick,rotate=#2] (#1) ++(12pt,3pt)--++(0,-6pt) --++(-2.5pt,0) 
++(2.8pt,6pt)-- ++(-2.5pt,0pt);
}

\usepackage{fancyhdr}
\pagestyle{fancy}
\lhead{University of Brasília}
\rhead{\thepage}
\cfoot{Building an H Bridge}
\renewcommand{\headrulewidth}{0.4pt}
\renewcommand{\footrulewidth}{0.4pt}

\begin{document}
    \title{Building an H Bridge}
    
    \author{\IEEEauthorblockN{Bruno H. F. Macedo and
            Gabriel F. P. Araújo}
        \IEEEauthorblockA{
        % \\Laboratório de Automação e Robótica - LARA
        \\Universidade de Brasília - UnB\\
        Brasília - DF - Brasil}
        }
    
    \maketitle
    
    \begin{abstract}
        
        This work presents the design and implementation of an H bridge. The circuit was designed to be used with low power DC motors. This report will present as well the design, prototyping and manufacturing of the board.
                
    \end{abstract}
    
    \begin{IEEEkeywords}
        H bridge, robotics, electric circuits, electronic circuits.        
    \end{IEEEkeywords}
    
 
    \section{\textbf{Introduction}}\label{sec:1}
	

	H bridges (see Figure~\ref{fig:bridge}) are electronic circuits used to enable a voltage to be aplied across a motor in either direction. These motor are often used in robotics to able a motor to run forward and backward.

\begin{figure*}[t]
\centering
% \begin{subfigure}[b]{0.4\textwidth}
	\centering%
	\includegraphics[height=.25\textwidth]{img/H_bridge.png}
	\caption{H Bridge figure borrowed from wikipedia~\cite{WIKI}.}\label{fig:bridge}%
% \end{subfigure}\hfill
\end{figure*}

\begin{figure}[htp!]
\begin{center}
\begin{circuitikz} %\label{sec:1}
    \draw (2,0) node[npn](q1) at (2,0){}
    (q1.E) node[npn](q4) at (2,-1.55){}
    (q1.E) -- (q4.C);
    \draw (-0.55,0) to[R] (q1.B){};
    \draw (-0.55,-1.55) to[R] (q4.B){};
    \draw (5,0) node[npn,xscale=-1](q3) at (4,0){}
    (q3.E) node[npn,xscale=-1](q2) at (4,-1.55){}
    (q3.E) -- (q2.C)
    (q1.C) -- (q3.C)
    (q4.E) -- (q2.E);
    \draw (3,-2.55) node[ground](gnd){}
    (q2.E) |- (gnd);
    \draw (q3.B)  to[R] (6,0);
    \draw (q2.B) to[R] (6,-1.55);
    % \draw (q1.E) node[elmech]{M}) (q3.E);
    \draw (-0.55,0) -- (-1.0,0) node[circ]{} -- (-1.0,-4) -- (7,-4) -- (7,-1.55) -- (6,-1.55);
    \draw (-0.55,-1.55) -- (-0.80,-1.55) -- (-0.8,-3.5) --(6.5,-3.5) -- (6.5,0) node[circ]{} -- (6,0);
    \draw (3,2) node[spdt,xscale=-1] (s1){}
    (s1.in) |- node[circ]{} (q1.C)
    (s1.out 1) |- (3.5,2.5) |- (7,2.5) |- (6,0)
    (s1.out 2) |- (-1,1.685) |- (-1,0);
    \draw (q1.E) to[sV, color=white, name=M1] (q3.E);
    \mymotor{M1}{0}
    \draw (q2.E) node[circ]{} |- (7.5,-2.31) to[battery] (7.5,1) |- (4,1) |- node[circ]{} (q3.C);
    % \draw (q1.C) --++(0,0.5) node[vcc]{+4.5\,\textnormal{V}};
    % \mymotor{M}{0}
    % \draw (q1.B) -- to[R=100<\ohm>]{};
    % \draw (0,0) node[npn](npn) at (0,0){};
\end{circuitikz}
\caption{Ponte H}\label{fig:schem}%
\end{center}
\end{figure}


    \section{\textbf{Theory}}\label{sec:2}
    Performing a circuit analysis~\cite{IRWIN} for the circuit in Figure~\ref{fig:schem_pspice} for $R_{b}$ being the base resistance, $V_{s}$ the source voltage and $R_{m}$ the equivalent motor resistance we obtain as follows:
\begin{itemize}
    
\item Considering Q1 and Q2 as short circuits and Q3 and Q4 as open nodes
\begin{equation}
I_{m} = \frac{V_{s}}{R_{m}}
\end{equation}
\item Considering Q3 and Q4 as short circuits and Q1 and Q2 as open nodes
\begin{equation}
I_{m} = -\frac{V_{s}}{R_{m}}
\end{equation}
\end{itemize}
    \section{\textbf{Simulation}}\label{sec:3}

    The circuit was simulated using PSpice software distributed by OrCAD. The circuit shown in Figure~\ref{fig:schem} was reproduced in PSpice (see Figure~\ref{fig:pspice}). The software does not support the transistor model used in the actual implementation, the TIP31C, instead the Q2N2222 was used to replace it given they are very similar, only the maximum supported values for voltages change. The value of all transistors base resistances was set to $100\:\Omega$. The DC motor simplified equivalent circuit is an inductor and a resistor~\cite{CHAPMAN}. The U8 and U7 times were set to $500\; ms$.\newline
    The motor used in the tests are the one included on the Magician Chassis\footnote{https://www.sparkfun.com/products/retired/12866} by Sparkfun, they have the following characteristics:
    \begin{itemize}
    \item Max Motor Voltage: 6 VDC;
    \item No Load Speed: $90\pm10$ rpm;
    \item No Load Current:190 mA (max.250 mA);
    \item Torque: 800 gf.cm;
    \end{itemize}
	Finally a transient analysis over the circuit was made and the current flowing through the motor is plotted in Figure~\ref{fig:plot_ponteh}. The simulated circuit behavior is exactly the expected, when Q1 and Q3 are activated the current flows... \todo{{FIND THE ROTATION DIRECTION FOR EACH TRANSISTOR PAIR}}

\begin{figure}
\centering
\includegraphics[height=.4\textwidth]{img/schem_pspice.png}
\caption{Schematic on PSpice program.}\label{fig:pspice}%
\end{figure}
	
\begin{figure}
\centering
\includegraphics[height=.3\textwidth]{img/plot.png}
\caption{Current x Time} \label{fig:plot_ponteh}
\end{figure}	

\begin{figure}[htp!]
\begin{center}
\begin{circuitikz} %\label{sec:1}
    \draw (2,0) node[npn](q1) at (2,0){}
    (q1.E) node[npn](q4) at (2,-1.55){}
    (q1.E) -- (q4.C);
    \draw (-0.55,0) to[R] (q1.B){};
    \draw (-0.55,-1.55) to[R] (q4.B){};
    \draw (5,0) node[npn,xscale=-1](q3) at (4,0){}
    (q3.E) node[npn,xscale=-1](q2) at (4,-1.55){}
    (q3.E) -- (q2.C)
    (q1.C) -- (q3.C)
    (q4.E) -- (q2.E);
    \draw (3,-2.55) node[ground](gnd){}
    (q2.E) |- (gnd);
    \draw (q3.B)  to[R] (6,0);
    \draw (q2.B) to[R] (6,-1.55);
    % \draw (q1.E) node[elmech]{M}) (q3.E);
    \draw (-0.55,0) -- (-1.0,0) node[circ]{} -- (-1.0,-4) -- (7,-4) -- (7,-1.55) -- (6,-1.55);
    \draw (-0.55,-1.55) -- (-0.80,-1.55) -- (-0.8,-3.5) --(6.5,-3.5) -- (6.5,0) node[circ]{} -- (6,0);
    \draw (3,2) node[spdt,xscale=-1] (s1){}
    (s1.in) |- node[circ]{} (q1.C)
    (s1.out 1) |- (3.5,2.5) |- (7,2.5) |- (6,0)
    (s1.out 2) |- (-1,1.685) |- (-1,0);
    \draw (q1.E) to[sV, color=white, name=M1] (q3.E);
    \mymotor{M1}{0}
    \draw (q2.E) node[circ]{} |- (7.5,-2.31) to[battery] (7.5,1) |- (4,1) |- node[circ]{} (q3.C);
    % \draw (q1.C) --++(0,0.5) node[vcc]{+4.5\,\textnormal{V}};
    % \mymotor{M}{0}
    % \draw (q1.B) -- to[R=100<\ohm>]{};
    % \draw (0,0) node[npn](npn) at (0,0){};
\end{circuitikz}
\caption{Ponte H}\label{fig:schem}%
\end{center}
\end{figure}


    \section{\textbf{Manufacturing}}\label{sec:4}
    For this project two PCB were made, hence in the presentation will be used two motors.
    As the simulation (see Section~\ref{sec:3}) proved that the designed circuit works as intended the physical components were assembled in a breadboard for further testing and validation before the final manufacturing \todo{complete with relevant information about final board}. Figure~\ref{fig:proto_h} shows the circuit properly assembled. This circuit's evaluation will be further discussed in Section~\ref{sec:5}.

    \subsubsection{Fritizing} % (fold)
    \label{ssub:fritizing}
        For aid the design of the printed circuit board it was used the software Fritizing\footnote{\url{http://fritzing.org/home/}}, that is an open source software used worldwide by hobbyists and it is an Electronic Design Automation software
        The Fritizing has 3 environments, shown in Figure~\ref{fig:breadboard}, they are breadboard, schematic and PCB, it was used only the schematic and design environments since the breadboard already has been evaluated.
    \todo{Show the fritzing environment, the schematics, board design and board manufacturing}
    % subsubsection fritizing (end)

    \subsubsection{Design} % (fold)
    \label{ssub:printing}

    % subsubsection printing (end)

    \subsubsection{Printing} % (fold)
    \label{ssub:printing}
    
    % subsubsection printing (end)
	\todo{TALK ABOUT CIRCUIT PRINTING}
	
\begin{figure}[t]
\centering
\centering%
\includegraphics[height=.40\textwidth]{img/FritzingBreadBoard.png}
\caption{Fritizing showing the breadboard environment.}
\label{fig:breadboard}%
\end{figure}


\begin{figure}
\centering

\begin{subfigure}{.45\columnwidth}
\includegraphics[height=4cm]{img/compontentes4.jpg}
\caption{Components in the PCB.}
\label{fig:pcb_top}
\end{subfigure}
\begin{subfigure}{.45\columnwidth}
\centering
\includegraphics[height=4cm]{img/solda_ja_saiu_da_jaula.jpg}
\caption{Connections in the PCB.}
\label{fig:pcb_bot}
\end{subfigure}
\begin{subfigure}{\columnwidth}
\centering
\includegraphics[height=4cm]{img/h_bridge_proto_close.jpg}
\caption{Circuit assembled on a protoboard.}
\label{fig:proto_h}
\end{subfigure}
\caption{Circuit assembling.}
\end{figure}

    \section{\textbf{Evaluation}}\label{sec:5}

	First of all the breadboard circuit (Figure~\ref{fig:proto_h}) was tested using a multimeter to ensure it's proper functionality. With Q1 and Q2 activated the voltage measured across the motor was $3.5\;V$, for future analysis this situation (Q1 and Q2 active) will be called S1. With Q3 and Q4 activated the voltage measured was $-2.6\;V$, this will be the S2 situation. The polarity inversion was accomplished, however the magnitude for both cases which was expected to be equal was in fact different. A slightly tension drop occurred when the current flew backwards \todo{explain why. idk why}. One important thing to mention is that when the transistor is turned off (base voltage = 0) it's output is not exactly equals zero, it shows a remainder voltage of $0.6\;V$ however this tension is not enough to move the motor.
	
	Despite the previous mentioned discrepancy the circuit performed as intentioned therefore the motor was plugged into the circuit. For S1 the motor spun in the counterclockwise direction. On the other hand for S2 it spun clockwise slightly slower, as a result of the voltage drop previous mentioned.
	
	Finally with the printed circuit board in hands the tension across the output pins were once again measured with a voltmeter. In case S1 the voltage showed up to be $3.6\;V$ which is equal to the breadboard circuit, the same occurred for S2. As the PCB performed exactly as the breadboard circuit the motor was connected and the final configuration was achieved. \todo{ADDITIONAL: MEASURE MOTOR SPIN SPEED}


    \section{\textbf{Conclusion}}\label{sec:6}

    % \input{Future_Works}
    
    %\nocite{*}
    \bibliographystyle{IEEEtran}
    \bibliography{Bibliography}
    % \toDo{Adequar as referências ao padrão do IEEE: \url{http://www.ieee.org/documents/ieeecitationref.pdf}. Quando
    % há 3 ou mais autores colocar et al.}
    
    
    
\end{document}
