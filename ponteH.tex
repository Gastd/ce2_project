\begin{figure}[htp!]
\begin{center}
\begin{circuitikz} %\label{sec:1}
    \draw (2,0) node[npn](q1) at (2,0){}
    (q1.E) node[npn](q4) at (2,-1.55){}
    (q1.E) -- (q4.C);
    \draw (-0.55,0) to[R] (q1.B){};
    \draw (-0.55,-1.55) to[R] (q4.B){};
    \draw (5,0) node[npn,xscale=-1](q3) at (4,0){}
    (q3.E) node[npn,xscale=-1](q2) at (4,-1.55){}
    (q3.E) -- (q2.C)
    (q1.C) -- (q3.C)
    (q4.E) -- (q2.E);
    \draw (3,-2.55) node[ground](gnd){}
    (q2.E) |- (gnd);
    \draw (q3.B)  to[R] (6,0);
    \draw (q2.B) to[R] (6,-1.55);
    % \draw (q1.E) node[elmech]{M}) (q3.E);
    \draw (-0.55,0) -- (-1.0,0) node[circ]{} -- (-1.0,-4) -- (7,-4) -- (7,-1.55) -- (6,-1.55);
    \draw (-0.55,-1.55) -- (-0.80,-1.55) -- (-0.8,-3.5) --(6.5,-3.5) -- (6.5,0) node[circ]{} -- (6,0);
    \draw (3,2) node[spdt,xscale=-1] (s1){}
    (s1.in) |- node[circ]{} (q1.C)
    (s1.out 1) |- (3.5,2.5) |- (7,2.5) |- (6,0)
    (s1.out 2) |- (-1,1.685) |- (-1,0);
    \draw (q1.E) to[sV, color=white, name=M1] (q3.E);
    \mymotor{M1}{0}
    \draw (q2.E) node[circ]{} |- (7.5,-2.31) to[battery] (7.5,1) |- (4,1) |- node[circ]{} (q3.C);
    % \draw (q1.C) --++(0,0.5) node[vcc]{+4.5\,\textnormal{V}};
    % \mymotor{M}{0}
    % \draw (q1.B) -- to[R=100<\ohm>]{};
    % \draw (0,0) node[npn](npn) at (0,0){};
\end{circuitikz}
\caption{H Bridge}\label{fig:schem}%
\end{center}
\end{figure}
